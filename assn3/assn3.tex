\documentclass{article}

\title{CS 452: Assignment 3}
\date{October 8, 2018}
\author{Rofael Aleezada}

\usepackage[left=2.5cm, right=2.5cm, top=2.5cm]{geometry}
\usepackage{mathtools, graphicx, amsmath, xfrac, algpseudocode}

\begin{document}
	\maketitle
	
	\section{Question 1}
	\begin{algorithmic}
		\Function{GreedyCoins}{amount, denominations}
			
			\Call{reverseSort}{denominations}
			
			remaining = amount
			
			coins = []
			
			\While{remaining $!=$ 0}
				\If{remaining $<$ 0}
					\State \Return{None}
				\EndIf
				\For{coin : denominations}
					\If{remaining - coin $>$ 0} 
						\State \Call{append}{coins, coin} 
						\State remaining -= coin
					\EndIf
				\EndFor
			\EndWhile
			
			\Return coins
		\EndFunction
	\end{algorithmic}

	Case where greedy algorithm fails:
	\begin{itemize}
		\item Let $amount = 30$ and $denominations = [20, 15, 1]$
		\item The most optimal solution is $[15, 15]$
		\item But our greedy algorithm will return $[20, 1, 1, 1, 1, 1, 1, 1, 1, 1, 1]$
	\end{itemize}

	\section{Question 2}
	\includegraphics[scale=0.1]{q2.jpg}
	\pagebreak
	
	\section{Question 3}
	\begin{center}
		Balanced Binary Tree \\ 
		\includegraphics[scale=0.5]{q3-bt.png} \\
		\textbf{Bits Needed: 3 (F = 101)}
	\end{center}
	\begin{center}
		Hoffman Tree \\
		\includegraphics[scale=0.5]{q3-huff.png} \\
		\textbf{Bits Needed: 5 (B = 11111)}
	\end{center}
	\pagebreak
	
	\section{Question 4}
	\begin{center}
		\includegraphics[scale=0.5]{q4.png}
	\end{center}

	\section{Question 5}
	The optimal Huffman encodings are
	\begin{itemize}
		\item A = 1111111
		\item B = 1111110
		\item C = 111110
		\item D = 11110
		\item E = 1110
		\item F = 110
		\item G = 10
		\item H = 0
	\end{itemize}
	Thus, to find the encoding of the $ith$ Fibonacci number in a Huffman tree with frequencies of the first n Fibonacci numbers
	\begin{itemize}
		\item H(nth character) = $0$
		\item H(ith character) = $H(i-1) + 2^{(n - i + 1)}$ $(1 < i < n)$
		\item H(1st character) = $H(2) + 1$
	\end{itemize}

	\section{Question 6}
	\textbf{False} \\ 
	\begin{center}
		Consider the following graph \\
		\includegraphics[scale=0.5]{q6.png} \\
		In this case, the shortest path between A and D is $A->D$. \\
		However, when all weights are doubled, the shortest path becomes $A->B->C->D$.
	\end{center}
	
	
	\section{Question 7}
	\textbf{False} \\ 
	Djikstra determines whether to update a cost and parent by comparing $cost(A) + dist(A,B) <= cost(B)$ \\
	With negative costs, this can result in an incorrect shortest path 
	\begin{center}
		\includegraphics[scale=0.5]{q7.png} \\ 
		Djikstra will return $B->C$, even though $B->A->C$ is optimal.
	\end{center}

	
	\section{Question 8}
	\textbf{True} \\
	MST depends on the ordering of the weights, rather than their actual values. Since the ordering will be the same, the MST will still be valid.	
	\pagebreak
	
	\section{Question 10}
	\begin{algorithmic}
		\Function{happyPeople}{people, candies}
			\State happy = 0
			\State max = 0
			\For{candy : candies}
				\If{candy $>$ max} 
					max = candy
				\EndIf
			\EndFor
			
			\State count = max * []
			\For{candy : candies}
				count[candy] += 1
			\EndFor
			
			\For{greed : people}
				\For{i .. [greed, max]}
					\If{count[i] $>$ 0}
						\State happy += 1 
						\State count[i] -= 1 
						\State break
					\EndIf
				\EndFor
			\EndFor
				
			\Return happy
		\EndFunction
	\end{algorithmic}
	
	\section{Question 11}
	\begin{algorithmic}
		\Function{randomPartition}{A, l, r} 
			\State rand = \Call{random}{l, r} \\
			\State temp = A[l] 
			\State A[l] = A[rand] 
			\State A[rand] = temp \\
			\State \Return \Call{partition}{A, l, r}
		\EndFunction \\
		
		\Function{randomQuickSort}{A, l, r}
			\If{l $<$ r}
				\State p = \Call{randomPartition}{A, l, r}
				\State \Call{randomQuickSort}{A, l, p -1}
				\State \Call{randomQuickSort}{A, p + 1, r}
			\EndIf
		\EndFunction \\
	\end{algorithmic} 

	Every call to randomPartition takes $O(1) + O(r - l)$, which is $O(n)$. \\
	Thus, each call in randomQuickSort is $T(n) = 2T(n/2) + O(n)$. \\
	Using the Master Theorem $(a = 2, b = 2, f(n) = O(n)$, \\
	we conclude that randomized Quick Sort is $O(n log n)$
	
	\section{Question 12}
	\begin{center}
		\includegraphics[scale=0.5]{q12_result.png} \\ 
		Median values will print when code is executed
	\end{center}
	
\end{document}
