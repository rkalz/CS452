\documentclass{report}

\title{CS452: Assignment 1}
\date{2018-09-04}
\author{Rofael Aleezada}

\usepackage{mathtools}

\begin{document}
    \maketitle

    \section{Question 1}
    Express the following in terms of Big-O notation:
        \subsection{$(n^3)/1000 - 100 * n^2 + 50$}
        	\begin{itemize}
        		\item{$O(n^3 / 1000 - 100 * n^2 + 50)$}
        		\item{$O(n^3 / 1000) - O(100*n^2) + O(50)$}
        		\item{$O(n^3) - O(n^2)$}
        	\end{itemize}
            The solution is $O(n^3)$.

        \subsection{$n^a + n^b$ ($a > b$ and $b > 0$)}
        	\begin{itemize}
        		\item{$O(n^a) + O(n^b)$}
        		\item{$a > b, n^a > n^b$}
        	\end{itemize}
            The solution is $O(n^a)$

    \section{Question 2}
    Assuming the Merge portion of merge sort remains $O(n)$, what would be the complexity of Merge sort if the array was divided into thirds instead of halves?
        \begin{itemize}
            \item{The part of merge sort where each half of the array is referenced recursively is $O(\log n)$}
            \item{This is technically $O(n\log_2 n)$, as an array of length $n$ will have $\log_2 n$ recursive layers due to dividing the array by two at each level}
            \item{Thus, if each recursive call divided the array into three pieces, that part becomes $O(\log_3 n)$}
            \item{Meaning this new merge sort is now $O(n\log_3 n)$, essentially $O(n\log n)$ and otherwise unchanged}
        \end{itemize}
        The solution is $O(n\log n)$

    \section{Question 3}
    Using the $O(n)$ Merge routine, what would be the runtime of a function that merges $k$ sorted arrays of length $n$
        \begin{itemize}
            \item{The runtime of performing Merge on two arrays of length n is $O(2n)$, $O(n)$}
            \item{For every $k$ rows, $k - 1$ merge operations are needed}
                \begin{itemize}
                    \item{2 arrays, 1 merges}
                    \item{3 arrays, 2 merges}
                    \item{4 arrays, 3 merges}
                \end{itemize}
            \item{Thus, merging k sorted arrays of length n would would have a runtime of $O((k-1)n)$}
            \item{This can be expanded to $O(kn - k)$, which simplifies to $O(kn)$}
        \end{itemize}
        The solution is $O(kn)$

    \section{Question 4}
    Extending Question 3, consider an algorithm that divides $k$ arrays, forms $k/2$ sorted length $2n$ arrays,
    and repeats until there is a single length $kn$ array
        \begin{itemize}
            \item{The complexity of the merge routine is still $O(n)$}
            \item{The major difference between this and the algorithm is in Question 3 is the number of merge operations}
                \begin{itemize}
                  \item{2 arrays, 1 split}
                  \item{4 arrays, 2 splits}
                  \item{8 arrays, 3 splits}
                \end{itemize}
            \item{For every k arrays, we perform $\log_2 k$ or $\log k$ splits}
            \item{In total, the complexity for this algorithm will be \textbf{$O(n\log k)$}}
        \end{itemize}
        The solution is $O(n\log k)$

    \section{Question 5}
    Arrange the given functions in order of increasing growth rate.
        \begin{enumerate}
            \item{$2^{\sqrt{\log_2 n}}$ is the slowest growing}
            \item{$\sqrt n$}
            \item{$10^{1.5}$}
            \item{$10^{5/3}$}
            \item{$10^n$ is the fastest growing}
        \end{enumerate}

    \section{Question 6}
    Given the array [3, 1, 2, 4, 5, 8, 7, 6, 9], which of the following are a potential pivot for Quicksort?
        \begin{itemize}
            \item{A pivot is defined as a position where all numbers before pivot are less than A[pivot] and all numbers after pivot are greater than A[pivot]}
            \item{Possible pivot values are 4 and 5}
        \end{itemize}
        The solutions are \textbf{values 4 and 5}

    \section{Question 7}
    Write a recurrence function for the running time of a recursive version of insertion sort.
    	\begin{itemize}
    		\item{Simply put, in insertion sort, you have to compare every item with every other item}
    		\item{Comparing a single value to all other values is $O(n - 1)$, and after a value is compared with all other values, the process is repeated $n-1$ times}
    		\item{$T(n) = n + (n - 1) + (n - 2) + ... + 2 + 1$}
    		\item{$T(n) = n + (n - 1) + (n-2) + ... + T(2)$}
    		\item{$T(n) = n + (n - 1) + T(n-2)$}
    		\item{The recurrence function of recursive insertion sort is $T(n) = T(n-1) + O(n)$}
		\end{itemize}
        The solution is $T(n) = T(n-1) + O(n)$

\end{document}