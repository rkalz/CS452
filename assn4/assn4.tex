\documentclass{article}

\title{CS 452: Assignment 3}
\date{October 8, 2018}
\author{Rofael Aleezada}

\usepackage[left=2.5cm, right=2.5cm, top=2.5cm]{geometry}
\usepackage{mathtools, graphicx, amsmath, xfrac, algpseudocode}

\begin{document}
	\maketitle
	
	\section{Question 1}
	\begin{center}
		\includegraphics[scale=1]{q1.png} \\
		The sum of odd numbers in this sequence is \textbf{92560705781}
	\end{center}

	\section{Question 2}
	\begin{center}
		\includegraphics[scale=0.75]{q2.png} \\
		The number of paths through a 20x20 maze is \textbf{538257874440}
	\end{center}
	
	\section{Question 3}
	\begin{center}
		\includegraphics[scale=0.75]{q3.png} \\
		The maximum total for traversing through triangle.txt is \textbf{7273}
	\end{center}

	\section{Question 4}
	\begin{center}
		\includegraphics[scale=0.75]{q4.png} \\
		The total possible ways to break the amount into change with those coins is \textbf{4935}
	\end{center}
	
	\section{Question 5}
	\begin{center}
		\includegraphics[scale=0.75]{q5.png} \\
		The largest possible sum for traversing matrix.txt is \textbf{310707}
	\end{center}

\section{Question 6}
\begin{center}
	\includegraphics[scale=0.75]{q6.png} \\
	There are \textbf{150} ways to roll 15 with 4 dice
\end{center}

\section{Question 7}
\begin{center}
	\includegraphics[scale=1]{q7.png} \\
	The 20th Catalan number is \textbf{6564120420}
\end{center}

\section{Question 8}
\begin{center}
	\includegraphics[scale=1]{q8.png} \\
	40 choose 5 is equal to \textbf{658008}
\end{center}
	
	
\end{document}
