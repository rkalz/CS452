\documentclass{article}

\title{CS 452: Assignment 2}
\date{September 15, 2018}
\author{Rofael Aleezada}

\usepackage[left=2.5cm, right=2.5cm, top=2.5cm]{geometry}
\usepackage{mathtools,graphicx}

\begin{document}
    \maketitle
    
    \section{Question 1}
    What is the recurrence of the running time for the closest-pair problem using the:
    	\begin{itemize}
    		\item Brute force algorithm
    			\begin{itemize}
    				\item The brute force algorithm is composed of iterative for loops
    				\item Given an array of points of length n, the outer loop will iterate n times (over each point)
    				\item The inner loop will run n - i times, where i is the current value of the outer loop
    				\item The longest run of the inner loop will be when i is zero, so the inner loop runs n times
    				\item Therefore, the total run time of the brute force algorithm is $O(n*n)$, or $O(n^2)$
    			 \end{itemize}
    		\item Divide and conquer algorithm
    			\begin{itemize}
    				\item The divide and conquer algorithm works as follows:
    				\item First, sort the points by their x-coordinates (an $O(nlogn)$ operation)
    				\item Then, divide the array of points along the median, and find the closest pair for each half
    				\item Then, compare the distances of the two halves and select the smaller of the two 
    				\item Continue until the recursion is complete, giving you the smallest distance
    				\item The recursive dividing of the array can be written as $T(n) = 2T(n/2)$
    				\item And the comparisons of all the distance values will be $O(n)$    			
    				\item Thus, the recurrence of the divide and conquer algorithm is $T(n) = 2T(n/2) + O(n)$
    			\end{itemize}
    	\end{itemize}
    
    \pagebreak
    \section{Question 2}
    When performing merge sort on the array $[5, 3, 8, 9, 1, 7, 0, 2, 6, 4]$, and if you were to glue together the two subarrays
    prior to the final merge, what would be the value of the seventh number in that array?
    	\begin{itemize}
    		\item The following graph shows how merge sort would sort this array
    	\end{itemize}
    	\begin{center}
    		\includegraphics[scale=0.5]{q2.png}
    	\end{center}
    	\begin{itemize}
    		\item As shown, the two arrays prior to the final merge are $[1, 3, 5, 8, 9]$ and $[0, 2, 4, 6, 7]$
    		\item \textbf{Assuming that the 1st element is 1, the 7th element is 2.} 
    	\end{itemize}
    
   \pagebreak
   \section{Question 3}
   Use Karatsuba's method to compute $1201 * 2430$, exiting the recursion when $n = 1$, and compare to the traditional method.
   Let $d$ be the number of digits in both numbers, $4$ in this case. 
   	\begin{itemize}
   		\item Karatsuba's method
   			\begin{center}
   				\includegraphics[scale=0.5]{q3.png}
   			\end{center}
   			\begin{itemize}
   				\item 12 operations are performed
   			\end{itemize}
   		\item Traditional method
   			\begin{itemize}
   				\item $1201 * 2430 = 1201 * 0 + 1201 * 30 + 1201 * 400 + 1201 * 2000$
   				\item $1201 * 0 = 1*0 + 0*0 + 2*0 + 1*0 = 0$ (4 operations)
   				\item $1201 * 30 = 10(1*3 + 0*3 + 2*3 + 1*3) = 36030$ (4 operations)
   				\item $1201 * 400 = 100(1*4 + 0*4 + 2*4 + 1*4) = 480400$ (4 operations)
   				\item $1201 * 2000 = 1000(1*2 + 0*2 + 2*2 + 1*2) = 2402000$ (4 operations)
   				\item $0 + 36030 + 480400 + 2402000 = 2918430$
   				\item 16 operations are performed
   			\end{itemize}
  	\end{itemize}
    	
    	

\end{document}